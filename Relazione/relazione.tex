\documentclass[a4paper,11pt]{article}
\usepackage[utf8]{inputenc}
\usepackage[italian]{babel}
%\usepackage[top=3 cm, bottom=3.5 cm, left=2.5 cm, right=2.5 cm]{geometry}
\usepackage{siunitx}
\usepackage{fancyhdr}
\usepackage{amsthm}
\usepackage{amsmath}
\usepackage{amssymb}
\usepackage{graphicx}
\usepackage{float}
\usepackage{braket}

\author{Ruggero Lot \\ ruggero90@gmail.com \\ fisica della materia \\}
\title{Esame di MBS}
\begin{document}
	\maketitle
	\paragraph{Problema:} % (fold)
	\label{par:problema}
		Si consideri un cristallo LJ con un interstiziale: ovvero una struttura
		FCC di N siti cristallini occupati da N+1 atomi, l' N+1-mo trovandosi in
		un sito non FCC. Si ponga l'atomo addizionale in modo da massimizzare la
		distanza dai primi vicini (il cristallo sarà sotto stress, si calcoli la
		struttura di equilibrio usando uno Steepest Descent).\\
		L'obiettivo è di calcolare il coefficiente di autodiffusione D, il quale,
		visto che il cristallo perfetto praticamente non diffonde, sarà dovuto 
		interamente all'atomo addizionale. Può essere che per vedere una diffusione
		apprezzabile sia necessario avvicinarsi alla T di fusione, facendo
		attenzione a verificare che il sistema rimanga solido (calcoli affini
		nello stato liquido possono essere d'aiuto). A questo fine può essere
		utile partire da T relativamente basse per poi eseguire simulazioni a T
		via via più alte, ogni volta partendo dallo stato finale della simulazione
		precedente.\\
		Eseguire un fit di $D(T)$ seguendo la relazione di Arrhenius
		\begin{equation*}
			D = D_0 e^{\frac{Q}{kT}}. 
		\end{equation*}
		Stimare la dipendenza di D da N per almeno una T.
	% paragraph problema (end)
	\newpage
	\tableofcontents
	\newpage
	\section{Software per la creazione del Sample} % (fold)
	\label{sec:software_per_la_creazione_del_sample}
		Il software ``starting\_positions'' è stato generato allo scopo di poter 
		generare un campione sul quale andare poi ad eseguire simulazioni per
		raccogliere dati sul comportamento delle particelle.\\
		Il campione generato attraverso questo strumento consiste in un reticolo
		FCC (di passo reticolare a piacere) con un atomo interstiziale(se 
		richiesto) contenuto in una scatola cubica (di lato l) e pronto per
		essere utilizzato in algoritmi capaci di implementare le condizioni periodiche al contorno.\\
		Il software consta di 2 algoritmi: il primo è quello di posizionamento 
		delle particelle all'interno della scatola, operazione che viene svolta
		in maniera sequenziale inserendo i 4 atomi per ogni cella unitaria nelle 
		posizioni FCC con la base:
		\begin{align}
			\mathbf a_1  &= a(0,0,0) & 
			\mathbf a_2  &= a(0,0.5,0.5) &
			\mathbf a_3  &= a(0.5,0,0.5) 
		\end{align}
		scartando eventuali atomi esterni alla scatola o doppi a causa delle pbc.\\
		Il secondo invece si occupa di portare il sistema, se sotto stress, ad uno 
		stato di equilibrio. Per risolvere questo problema è stato utilizzato uno 
		Steepest Descent
	% section software_per_la_creazione_del_sample (end)
\end{document}